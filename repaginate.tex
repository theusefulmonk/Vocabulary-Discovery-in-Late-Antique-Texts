\documentclass[letterpaper]{article}

%%%%CREDITS%%%%%
%% https://askubuntu.com/questions/32048/renumber-pages-of-a-pdf
%% https://tex.stackexchange.com/questions/117960/header-and-page-numbers-with-pdfpages
%%%%%%%%%%%%%%%%

%% Notes begin with a double comment character: %%. Unused macros and commands, however, begin with a single comment character: %

\usepackage[letterpaper]{geometry} %% Change paper format for easier reading on a small device, such as an iPhone. a5paper fits well on an iPad mini.

\usepackage{pdfpages} %% This package includes pages from your scanned source (or from any pdf file you choose).

%% The key is to use hyperref with the pdfpagelabels option. hypperref uses \thepage macro to set the actual pdf label in the document's metadata.
\usepackage[pdfpagelabels]{hyperref} %% Must be loaded last or as late as possible because it redefines other commands.


\begin{document}

%% The pattern below is simple. For each section of the final 
%% document needing distinct pagination, first redefine the \thepage 
%% macro, then include the relevant sections of the source pdf. Repeat 
%% as many times as necessary. Optionally, if your text includes only 
%% part of the original, such as a compilation of book chapters, you 
%% can reset the counter at any of these stages so that it starts at 
%% whatever arbitrary number you want. In this way, it is possible to 
%% start numbering, for example, at p. 345 after including a cover 
%% page with notes or bibliographic information.
 
%% Any commands not needed for a particular repagination task can be 
%% "commented out:" that is, made inoperable by prefixing the text with a % 
%% character. 

%% The order in which these commands are applied matters. Rearrange sections 
%% by copying and pasting as needed to produce the desired result.

%% For one or more coverpages. Comment out if unneeded.
\renewcommand*{\thepage}{Cover-\arabic{page}}
\includepdf[pages=240]{sources/hdn-ocr-optimized.pdf} %% Filename must be without spaces.

%% You can also add your own titlepage as a cover if you like. In that case, 
%% comment out the \includepdf instruction and add instead the following:
%\begin{titlepage}
%\title{La Théologie de Jacques de Saroug}
%\maketitle
%\begin{center}
%% Put bibliographic information, or anything you want, here.
%\end{center}
%\end{titlepage}

%% For frontmatter with lowercase roman numerals. Comment out if not needed 
%% (such as is usually the case for journal articles).
\renewcommand*{\thepage}{\roman{page}} %% Use \Roman for upper case Roman numerals.
\setcounter{page}{1} %% Set the counter to whatever arbitrary value
\includepdf[pages=1-14]{sources/hdn-ocr-optimized.pdf} %% Filename must be without spaces.
%% for body with regular arabic numerals
\renewcommand*{\thepage}{\arabic{page}}
\setcounter{page}{1} %% Set the counter to whatever arbitrary value
%% you want. For a book, it's usually 1
\includepdf[pages=15-]{sources/hdn-ocr-optimized.pdf} %% Filename must be without spaces.

\end{document}
